\documentclass[11pt,article,oneside]{memoir}
\usepackage{org-preamble-xelatex}
%\input{vc}



\title{\bigskip \bigskip Introduction}
 
\author{\bigskip\Large }

\begin{document}  

\setsansfont[Mapping=tex-text, BoldFont={* Bold SemiCondensed}, ItalicFont={* Semibold SemiCondensed Italic}]{Myriad Pro}
\setmonofont[Mapping=tex-text,Scale=MatchLowercase]{Consolas}
\setromanfont[Mapping=tex-text,Numbers=OldStyle]{Minion Pro}


\setkeys{Gin}{width=1\textwidth}  
\setromanfont[Mapping=tex-text,Numbers=OldStyle]{Minion Pro} 
\setsansfont[Mapping=tex-text]{Minion Pro} 
\setmonofont[Mapping=tex-text,Scale=0.8]{Consolas}
\chapterstyle{article-4} 
\pagestyle{kjh}

\published{Draft only. Please do not cite without permission.}

\maketitle


\begin{quote}
The underlying principle of conservation has been described as the
application of common sense to common problems for the common good. If
the description is correct, then conservation is the great fundamental
basis for national efficiency. In this stage of the world's history to
be fearless, to be just, and to be efficient are the three great
requirements of national life. National efficiency is the result of
natural resources well handled, of freedom of opportunity for every man,
and of the inherent capacity, trained ability, knowledge and will,
collectively and individually to use that opportunity."

-- President Theodore Roosevelt (1909)
\end{quote}

\begin{quote}
``The post-war expansion of our population and of our commerce and
industry has placed new demands on the use of public lands; so has the
increased leisure time of our people and the resultant demands for
recreation areas . . . we should dedicate our lands and the resources
thereof to accomplish the maximum good for the maximum number for the
longest period of time.''

-- Senator Wayne Aspinall (1963)
\end{quote}

Looking back on the four decades since William Hewlett and Richard
Packard founded their high technology firm in a Palo Alto garage, the
\emph{San Jose Mercury}, the city's oldest newspaper, through a series
of articles surveyed ten causes of Silicon Valley's success. One of
those causes was environmental: water. Beginning in 1965, water from the
Sacramento-San Joaquin River Delta flowed forty-four miles through the
South Bay Aqueduct. Sig Sanchez, who served as the director of the Santa
Clara Valley Water District by the 1990s, recalled the project's
importance to Silicon Valley. ``It was our first import project, and it
was essential to the growth we were anticipating,'' he remarked. ``The
valley would have grown, but not at the rate it has, because we would
not have been able to accommodate the Silicon Valley.''\footnote{``Water
  made orchards, silicon chip industry sprout faster,'' \emph{San Jose
  Mercury}, December 22, 1999.}

Three water projects would eventually bring a consistent source of fresh
water into Santa Clara Valley. Twenty years after the South Bay
Aqueduct, the San Felipe water system would be constructed through
Pacheco Pass to move water out of the San Joaquin Valley. Additionally,
the Valley consumed water from the Hetch Hetchy system, which today
supplies about half the water used throughout the Bay. Additional water
conservation projects, including dams, reservoirs, percolation ponds,
canals, and runoff capture, together helped provide the Valley with its
water needs.

James Lenihan, who served on the water conservation board in the 1960s
and later served on the board for the Santa Clara Valley Water District,
credited water project success on the absence of an environmental
movement. ``With the environmental movement, we would never have been
able to do now what we did then,'' he said. ``We were able to get it in
place before the movement came along to say no more reservoirs, no
pipelines, no aqueducts.''\footnote{``Water made orchards, silicon chip
  industry sprout faster,'' \emph{San Jose Mercury}, December 22, 1999.}

But the environmental movement was active, and resist they did.
Unrestricted urban growth and economic development characterized much of
Santa Clara Valley in the latter half of the twentieth century, but
little told is the story of the environmental consequences of Silicon
Valley.

The story told here stands at the intersections of postindustrial
society, environmental history, and the spatial politics of
suburbanization and economic development. City leaders, residents,
laborers, migrant workers, and others all experienced rapid urban change
after World War II and responded in various ways to these changes. One
consequence of these changes is the enormous demands made upon water
resources in the Far West.

Silicon Valley has only just begun to attract attention of historians.
Margaret O'Mara and John Findley have written the best accounts of
Silicon Valley and its place in urban processes. But their stories focus
on the institutional and cultural construction of Silicon Valley.

{[}Silicon Valley. The Santa Clara Valley lies between the Santa Cruz
Mountains on the west and the Diablo Range to the east, largely
encompassing the municipalities of San Jose, Mountain View, Sunnyvale,
Palo Alto, Menlo Park, and Redwood City. Between 1945 and 2000, San Jose
became the largest city in the Bay Area and ranked the near the top of
population growth in the nation.{]}{[}/figures/sv.png{]}

Metropolitan growth, annexations, and incorporations were so rapid taht
Robert Self called the process a ``land rush.''\footnote{Self,
  \emph{American Babylon}, 120.}

The study approaches political history at the local level. City leaders,
activists, residents, and so on shaped the implementation and
interpretation of policies \ldots{}

In examining the human and environmental costs of water, electronics,
outdoor recreation, urban development, and wilderness areas, this study
evaluates the costs of these programs. Although this study is tightly
focused on a specific region, it has greater bearing on understanding
the inherent tension between local and national politics. In the Santa
Clara Valley, the development of new landscapes, resulting from various
perceptions of the region as farmland, electronics manufacturer, urban
oasis, and tourist destination shaped the use of water in the valley.

Water and technology are icons in California. The battle over the Hetch
Hetchy reservoir and the growth of environmentalism calls to mind the
importance of water and its battles in the Far West, while the
electronics industry has located its origin myth in the former orchard
fields surrounding Stanford University. These two icons share the
California landscape in ways not yet examined by historians. Although
considered separate landscapes, they are entangled in ways not
previously understood. Water and the Santa Clara Valley share what
Richard White has called a hybrid landscape. Neither rural or urban,
hybrid landscapes are complex creations of natural and cultural systems
that shape a place.⁠\footnote{Richard White, ``From Wilderness to Hybrid
  Landscapes: The Cultural Turn in Environmental History,'' \emph{The
  Historian} 66 (September 2004): 562-664.}

Water became a linchpin in contests over power conflicts: who controlled
water, who had access to water, how water should and could be used. The
valley's agricultural past required access to water to support orchard
fields. Later in the twentieth century, water became key to
semiconductor manufacturers, who used an ultra-purified water in the
process of manufacturing electronics components. Supporting urban growth
offered a third theme in contests over water, as the valley's population
exploded in the mid-twentieth century. And finally, residents new and
old made demands on water as recreation. Dam projects by the state of
California in and around the Santa Clara Valley became tied to
recreation and tourism while simultaneously supporting other water
projects for manufacturing and urban growth. As water demands rose, it
also gave growth to a modern environmental movement that challenged
suburban sprawl, industrial development, and recreation in the Bay Area
while also breathing life into environmental movements across the
nation.

Efforts by federal regulators, city officials, business leaders, and
farmers all attempted to maximize their access to water. Beginning in
the 1930s, orchard fields in the Santa Clara Valley became national
producers of prunes, cherries, peaches, \{more\}. The growing
agribusiness presence in the Bay Area made demands on local water
resources and led federal officials and state political leaders to
pursue water and irrigation projects that would sustain farmlands. Yet
by the 1930s so much water was being pumped out of underground water
basins that Stanford geologists discovered Santa Clara Valley had sunk
four feet in just twenty years.\footnote{``Data Show Sinking of Bay
  Area,'' \emph{Los Angeles Times}, March 20, 1934, p.~1. The process of
  land sinking is known as subsidence and results from the weight of
  land compacting underground sand and gravel aquifers that have been
  drained. Subsidence would not end until 1969; by then, downtown San
  Jose had sunk ten feet and Alviso along the coastal bay had sunk
  thirteen feet.} Concerns over the Valley's access to water led to the
development of the Central Valley Project in the late 1930s.

South Bay Aqueduct and San Joaquin Valley Project, along with San
Francisco's Hetch Hetchy reservoir, supplied over half the water to the
Valley. The water projects were essential to support the growing
suburbs, high tech industries, and agriculture in the Valley. The story
here re-situates how we understand Silicon Valley by placing the
environment at the core of the story. \{EXPAND\}

\subsection{Water and the West}

Historians of western America have long identified water as a central
component in the region's historical experience. Frederick Jackson
Turner \ldots{}..

Walter Prescott Webb's classic The Great Plains (193?) revised Turner's
telling of Western history and located in the Plains a different, if not
outright hostile, environment --- an experience far different from the
environment Turner's Americans experienced in the East. For Webb, water
was a defining feature of the Great Plains: he bounded the region based
on its precipitation; he recounted the region's troubles with
agriculture and water use; he maintained the importance of irrigation
projects; etc. Having no trees and scarce water, the Great Plains
settlers

\begin{quote}
Were thrown by Mother Necessity into the clutch of new circumstances.
Their plight has been stated in this way: east of the Mississippi
civilization stood on three legs--land, water, and timber; west of the
Mississippi not one but two of these legs were withdrawn,--water and
timber,--and civilization was left on one leg--land. It is small wonder
that it toppled over in temporary failure.
\end{quote}

Webb's settlers turned to underground aquifers -- their only source of
substantial water -- and developed new methods for dealing with scarce
resources: irrigation systems, new legal systems for allocating water
rights. The prairie was ``a land of survival where nature has most
stubbornly resisted the efforts of man. Nature's very stubbornness has
driven man to the innovations which he has made.'' Struggle against a
resistant environment has been told by subsequent historians. James
Malin's more ecologically sophisticated studies of the Plains tells a
similar story of settlers moving from ``forest man'' to ``grass man.''
Paul Bonnifield's The Dust Bowl provides a similar narrative of humans
overcoming the environment.

\subsection{Postwar Western Demographics}

The mobilization of the country in World War II dramatically altered the
nation and people's daily lives. Following a decade of economic
depression, people migrated to metropolitan areas to take new wartime
industries. The American West especially felt the impact of this
demographic shift, leading Carl Abbott to remark that launched ``the
entire West into the half-century of head-long urbanization.''\footnote{Abbott,
  \emph{Metropolitan Frontier}, p.~4.} Six of the largest western
metropolitan areas in 1940 grew by 380\% over the course of fifty years,
compared to just 64\% for the six largest in the East.⁠\footnote{Abbott,
  \emph{Metropolitan Frontier}, p.~xiii.} Just as populations were
migrating to metropolitan centers, the number of people living inside
city boundaries was declining. The attraction of suburban life
\{expand\}

\subsection{The Bay Area}

The San Francisco Bay Area is among the most highly populated places in
the United States. The area is physiographically situated between the
Santa Cruz mountains to the west and the Diablo Range to the east, a
tightly compact area spanning only \{MILES\} across and \{MILES\} long.
Hydrologically, the region is fed by several rivers and aquifers
dominated by the \{FACT\}.

{[}Bay Area hydrology.{]}{[}/figures/hydrology.png{]}

The Mediterranean climate of the Bay Area means the region receives very
little rainfall outside of the period between November and April. Around
seven percent of California land in 1951 was agricultural, and around
half of that land required consistent irrigation.\footnote{Harry Blaney,
  ``Use of Water by Irrigated Crops in California,'' \emph{Journal of
  the American Water Works Association} 43 (March 1951): 189.} On
average, the semiarid Valley receives about 14 inches of rain
annually.\footnote{``Water made orchards, silicon chip industry sprout
  faster,'' \emph{San Jose Mercury}, December 22, 1999.}

\subsection{The Landscape of Conflict}

Geographer John Wright argued that ``places are best seen as shifting
stages where the exercise of power and resistance to it vie for
dominance.''

Questioning the need for water projects at all.

{[}Water waste{]}{[}/figures/money\_waste.jpg{]}

\subsection{Organization}

The dissertation is organized chronologically and thematically. Each
chapter is roughly chronological to one another, but thematic in their
narrative to explore the various processes and contests at work in the
Valley. The themes are organized around irrigation, manufacturing,
tourism, and urban growth.

\textbf{Chapter 1: The Western Water Landscape}

The water landscape

The urban landscape

The agricultural landscape

The recreational landscape

\textbf{Chapter 2: The Valley of Heart's Delight}

agriculture

\textbf{Chapter 3: ``Carved from a Forest of Fruit Trees'': Urban Growth
and Water Resources}

urban

Joe Ruscigno, a lifetime farmer and resident of the Valley, had started
a new occupation in 1952. ``Guess I've pulled out 150 acres of trees
since the first of the year,'' he told the San Francisco Chronicle.
Ruscigno lamented the uprooting of the prune trees to the bulldozer he
now sat upon, but ``what can you do? . . . The subdivisions were coming
in all around us and when they made me a good offer I sold
out.''⁠\footnote{``Santa Clara County -- Scene of the Big Boom,''
  \emph{San Francisco Chronicle}, May 11, 1952.} The Valley's
suburbanization led to similar experiences as those by Ruscigno.

\textbf{Chapter 4: How Clean is Clean?}

Industry

\textbf{Chapter 5: Leisure, Recreation, and Water}

recreation

Just as urban and business resources were making greater demands on the
Valley's water resources, a shift in thought about the utility of water
resources became centered on the experiential value of outdoor
recreational tourism. Recreation presented an opportunity to use water
that could continue to fuel monetary values into the Valley.\footnote{Marguerite
  S. Shaffer, \emph{See America First: Tourism and National Identity,
  1880-1940} (Washington, DC: Smithsonian Institution Press, 2001) and
  David Louter, \emph{Windshield Wilderness: Cars, Roads, and Nature in
  Washington's National Parks} (Seattle: University of Washington Press,
  2006). For a similar history of the national forests and the ways in
  which the American public interacted with them through activities such
  as skiing, hiking, and camping see Paul S. Sutter, \emph{Driven Wild:
  How the Fight Against Automobiles Launched the Modern Wilderness
  Movement} (Seattle: University of Washington Press, 2002), Tom Wolf,
  \emph{Arthur Carhart: Wilderness Prophet} (Boulder: University Press
  of Colorado, 2008), and Michael W. Childers, ``Fire on the Mountain:
  Growth and Conflict in Colorado Ski Country,'' Dissertation,
  University of Nevada, Las Vegas, May 2010. Marion Clawson and Burnell
  Held, \emph{The Federal Lands: Their Use and Management} (Baltimore:
  Resources for the Future, Inc., 1957), 341; Hal K. Rothman,
  \emph{Devil's Bargains: Tourism in the Twentieth Century American
  West} (Lawrence: University Press of Kansas, 1998), 23-25.}

\subsubsection{NOTES}

\begin{enumerate}
\def\labelenumi{(\arabic{enumi})}
\setcounter{enumi}{146}
\item
  ``massive military expenditures have spurred a major spatial shift in
  manufacturing production to the so-called defense perimeter. In this
  manner, the federal government, by means of its military budget and
  locational preferences, has promoted uneven regional development.''
\item
  ``At the center of these processes of economic structuring and spatial
  shifts is the information sector, which includes both information
  technologies (the hardware part) and the use of advanced information
  systems (the software part). The spatial consequences of these
  information technologies and systems deserve more attention from
  historians than they have received. Introduction of these new
  technologies in the manufacturing and service sectors, for example,
  directly affects where businesses choose to locate -- since it reduces
  the need for spatial proximity.''
\item
  Sees Orange County as a leader in these processes -- economic
  restructuring, militarization of production, spatial transformation,
  information technologies and systems, internationalization of a
  regional economy. --- Olin article (above)
\end{enumerate}

Santa Clara County and Salt Lake County came to dominate cultural and
commercial influence in the United States, becoming what Carl Abbott
called the ``new centers for American life.''⁠1


\end{document}
