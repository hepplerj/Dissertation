\documentclass[11pt,article,oneside]{memoir}
\usepackage{org-preamble-xelatex}
\input{vc}



 
\author{\bigskip\Large }

\begin{document}  

\setsansfont[Mapping=tex-text, BoldFont={* Bold SemiCondensed}, ItalicFont={* Semibold SemiCondensed Italic}]{Myriad Pro}
\setmonofont[Mapping=tex-text,Scale=MatchLowercase]{Consolas}
\setromanfont[Mapping=tex-text,Numbers=OldStyle]{Minion Pro}


\setkeys{Gin}{width=1\textwidth}  
\setromanfont[Mapping=tex-text,Numbers=OldStyle]{Minion Pro} 
\setsansfont[Mapping=tex-text]{Minion Pro} 
\setmonofont[Mapping=tex-text,Scale=0.8]{Consolas}
\chapterstyle{article-4} 
\pagestyle{kjh}

\published{Draft only. Please do not cite without permission.}



ABSTRACT

Watering the Valley: Growth, Conflict, and Urban Environmentalism in
Silicon Valley

By

Jason A. Heppler

Dr.~Patrick Jones, Examination Committee Chair Professor of History
University of Nebraska-Lincoln

This dissertation examines the environmental, economic, and cultural
conflicts over the development and use of water projects in California's
Santa Clara Valley between 1930 and 1990. Concerns over the nation's
water resources emerged in the latter half of the twentieth century,
particularly in western states like California. Far western states from
Arizona to Washington engaged in new water projects that sought to
distribute the resource equally among states, in an attempt to meet the
demands of farms, urban growth, and industrial development. Competing
ideas about water use and access proved increasingly problematic as
private and public interests jostled for access to scarce water
resources. The competing idea of preservation and sustainability helped
give rise to a modern environmental movement, one that increasingly
criticized urban and suburban growth, industrial toxics, and sought the
preservation of wilderness areas. The two emergent views over nature
conflicted over the issue of water resources, particularly regarding new
industrial sectors of high technology. The Santa Clara Valley emerged as
the cultural powerhouse in high technology, placing California at the
center of these conflicts. By the 1980s, growing numbers of critics were
denouncing the environmental impact unrestrained growth and computer
companies were placing on the Valley's water resources as political
battles raged over the environmental, social, and economic impact of
high technology. California voters, once happy to give approval to large
public water projects, began pushing back. Controversies such as
California voters' rejection of the Central Valley Water Project in
1973, the \_\_\_\_\_, the legal battles with Fairchild Semiconductor
regarding groundwater contamination, and the emergence of the Silicon
Valley Toxics Coalition pitted regional demands against new concerns
about the environmental and social consequences of commercial
development and urban growth. The story here not only tells the story of
California water and western water generally, but Americans' changing
understanding of nature and environmental costs that accompanies urban
and commercial development.


\end{document}
